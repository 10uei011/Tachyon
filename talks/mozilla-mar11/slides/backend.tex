% -- Slide ---------------------------------------------------------------------
\begin{frame}
\frametitle{\bf Backend Overview}

% TODO: diagram
\begin{itemize}
    \item Basic block ordering
    \item Instruction numbering and intervals
    \item Hints
    \item Operand assignment
    \item SSA deconstruction
    \item Translation
    \item Linking and assembly
\end{itemize}

\end{frame}
% ------------------------------------------------------------------------------

% -- Slide ---------------------------------------------------------------------
\begin{frame}
\frametitle{\bf Linker}

% used for static fn calls + strings
\end{frame}
% ------------------------------------------------------------------------------

% -- Slide ---------------------------------------------------------------------
\begin{frame}
\frametitle{\bf Register allocator}

\begin{itemize}
    \item Based on a modified linear scan allocator
    \item Extended the algorithm to allow hints
    \begin{itemize}
        \item Required to support intricacies of x86 architecture
    \end{itemize}
    \item 2 types of hints supported for a given position in the code:
    \begin{itemize}
        \item Register should be free at that point
        \item SSA variable should be assigned to a particular register at that point
    \end{itemize}
    \item Hints are weak properties and may not be respected, so code generation must enforce
    them when required
\end{itemize}
\end{frame}
% ------------------------------------------------------------------------------

% -- Slide ---------------------------------------------------------------------
\begin{frame}[fragile]
\frametitle{\bf Supported architectures}

\begin{itemize}
    \item Tachyon uses its own assembler for maximum flexibility
    \item Retargettable backend currently supports x86 and x86\_64
    \item Assembly code produced by a chain of calls that resemble ASM
    listings
\end{itemize}

\begin{block}<+->{Assembly framework example}
\begin{lstlisting}[language=]
    this.asm.

    mov(temp, ctxTemp).
    mov($(0), temp).
    
    mov($(argsRegNb), argPtr).
    sub(numArgs, argPtr).
    cmovl(temp, argPtr).
\end{lstlisting}
\end{block}
\end{frame}
% ------------------------------------------------------------------------------

% -- Slide ---------------------------------------------------------------------
\begin{frame}
\frametitle{\bf Calling protocol}

\begin{itemize}
    \item Stack pointer and context pointer have dedicated registers
    \item Return values are passed in a register (currently EAX)
    \item Up to $n$ first arguments passed using registers (currently, $n =
    4$)
    \item Caller-save protocol
    \item Callee pops the activation record to support tail call optimisations
\end{itemize}
\end{frame}
% ------------------------------------------------------------------------------

% -- Slide ---------------------------------------------------------------------
\begin{frame}
\frametitle{\bf Object representations}

\begin{itemize}
    \item Flexible object representation through JS layout objects
    \begin{itemize}
        \item Accessor functions dynamically generated for each layout
        \item 3 basic layouts: basic objects, functions and arrays
    \end{itemize}
    \begin{itemize}
        \item Basic objects contain a header and a map (pointer to a
        hashtable)
        \item Arrays also contain a pointer to an external table
        \item Functions also contain a pointer to mutable cells (to support
        closures)
    \end{itemize}
    \item Other objects are heap objects
    \begin{itemize}
        \item context objects, strings, etc.
    \end{itemize}
\end{itemize}

% header (32bits) contains type ID

% Objects
% - Functions/Closures (no diff)
% - Numbers - fixnums have 2 LSBs == 0
% - Strings
% - Arrays

\end{frame}
% ------------------------------------------------------------------------------

% % -- Slide ---------------------------------------------------------------------
% \begin{frame}
% \frametitle{\bf Memory management}
% 
% % - Allocator
% % - Executable code
% \end{frame}
% % ------------------------------------------------------------------------------

% -- Slide ---------------------------------------------------------------------
\begin{frame}
\frametitle{\bf Meta-circularity}

% - Standard library
% - Primitives
% - FFI as bridge
% - Memory access
%   - Pointers representation
%   - Memory block
% - Numbers
%   - Representation limits
\end{frame}
% ------------------------------------------------------------------------------

% -- Slide ---------------------------------------------------------------------
\begin{frame}
\frametitle{\bf Limitations of JS for compiling writing}

% - Infinite precision numbers
% - Absence of native hash tables, sets, lists, etc.
% - Bitwise operations limited to 32 bits
% - Allocation behavior of standard operations
% - Modules
% - I/O
% - Memory access
\end{frame}
% ------------------------------------------------------------------------------

% -- Slide ---------------------------------------------------------------------
\begin{frame}
\frametitle{\bf Experience}

% - Dynamicity and testing/refactoring
% - Metacircularity and intuitive understanding of the language usage patterns
\end{frame}
% ------------------------------------------------------------------------------

% -- Slide ---------------------------------------------------------------------
\begin{frame}
\frametitle{\bf Performance}
                          % Sunspider control flow recursive (200 iterations)
% - Compilation Time  :   0.8 s tachyon, 0.82 s v8 (combined) 
% - Execution Time    :   5.0 s tachyon,
% - Bootstrapping Time: 6 mins v8, ~1hr tachyon
%                       40-50 gb heap (no gc)
%                       v8 < 1 gb

% d8 version 3.2.3.1
\end{frame}
% ------------------------------------------------------------------------------

